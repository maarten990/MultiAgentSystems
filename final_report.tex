\documentclass[a4paper]{article}
\usepackage[english]{babel}
\usepackage{amsmath}

\title{Building Evacuation}
\author{Maarten de Jonge \\
        Edwin Odijk \\
        Roelof van der Heijden}

\begin{document}

\maketitle

\section{Introduction}
In this report we simulate people escaping from a burning building using the BDI framework in NetLogo. We create an environment with people and fire randomly distributed throughout the building. People randomly walk through the building until they spot fire, at which point they try to run towards the escape and alarm any others they meet along their way. The simulation ends when all people either escaped or died in a fire.

Using this simulation, we aim to find what affects the speed and survival rate of such an evacuation.

Additionally, we describe the NetLogo implementation, explaining the functions of buttons and how they are implemented.

\section{Method}
%stuff here?

\subsection{Environment}
The environment is a grid of squares with either floor, wall, fire or exit tiles. There is always at least one exit available, and fire may spread to other tiles. To ensure a level of consistency across tests, we use several preset wall layouts.

\subsection{Agents}
The agents in this simulation are people trying to escape from the building while avoiding fire. In the BDI framework, they have:
\begin{itemize}
\item Beliefs: Layout of the building, locations of fire, location of exit, locations of other agents
\item Desires: Roam (when no fire is spotted), Find exit, Escape, Alarm others
\item Intentions: Move towards exit, Move away from fire, Move towards other agent, Alarm other agent
\end{itemize}
Agents have the following properties:
\begin{itemize}
\item Whether they are aware of fire.
\item Whether they may/may not pass through eachother.
\item Detection radius for fire.
\item Communication radius to alarm other agents.
\item Whether they prioritize helping other or escaping.
\end{itemize}

When an agent spots fire, he becomes aware of the fire. Whenever they encounter another agent within their communication radius, they will inform them of all fire tiles they know.

\subsection{NetLogo Controls}
To provide more insight in the workings of this simulation, we provide a overview below of the controls implemented in NetLogo, as not all of them may be self-explanatory.

\subsubsection{Draw, Save and Clear walls}
Walls used by the simulation are handdrawn and saved in a separate txt file. Upon running Setup, the file is read to redraw the walls.

To use the function, first ensure that Setup is run at least once. If walls are already present, you can click clear\_walls (thereby removing the txt file) and run Setup again to redraw the environment with an empty sheet. Clicking draw\_walls allows you to start drawing with the environment. When you are finished, click draw\_walls again to stop drawing, then click save\_walls to save the new environment, which will now be redrawn every time Setup is run.

\subsubsection{Throttle Speed}
As the simulation may slow down at times, it may be preferrable to watch the simulation at a constant speed. Turning throttle\_speed on slows all steps to an equal speed of 100 milliseconds. This does mean that the overall speed of the simulation is slowed down, however, so this should be off when swift simulations are desired.

\subsubsection{Fire Spread Rate}
Controls the speed at which fire spreads. At every step, fire may spread to each neighbouring tile with a probability defined by fire\_spread\_rate in promille for each of these tiles.

\subsubsection{Vision}
The properties fire\_vision and person\_vision control the vision in which each agent can observe nearby fire or other agents respectively. This vision is represented by an invisible circle around the agent, with a radius equal to the value defined on the slider.

\section{Experiments}
\begin{itemize}
\item Can we prevent traffic jam behaviour?
\item How much do the following properties affect overall survival rate?
\begin{itemize}
\item Rate at which fire spreads
\item Whether agents have prior knowledge of exit locations
\item The ratio between agents that prioritize saving themselves over saving others
\end{itemize}
\end{itemize}

\subsection{Basic Setup}
The simplest form in which we can simulate this problem is with the following properties:
\begin{itemize}
\item One exit, of which all agents know the location
\item One randomly placed fire tile, which does not spread
\item All agents have the same priority, i.e. escape rather than going out of their way to find others
\item One simple wall between the agents and the exit, so they are forced to walk around it
\end{itemize}
For next week's prototype, we aim to build this basic setup.

\section{Results}

\section{Conclusion}

\end{document}